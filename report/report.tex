\documentclass[a4]{article}
\usepackage{hyperref}
\usepackage[swedish]{babel,varioref}
\usepackage[utf8]{inputenc}
\usepackage{graphicx}

\begin{document}
{
\Large
Solsystemet\\
Klassisk Mekanik\\
~\\
Benjamin Nauck, c05ben@cs.umu.se\\
Emil Eriksson, c07een@cs.umu.se\\
~\\
Försök två...
%\today{}
}
\clearpage

\section{Introduktion}
I följande laboration kommer vi att gå igenom svar på uppgifter vi fått och
diskussion om dessa.
Laborationen gick ut på att implementera och analysera n-kropps-problem med
hjälp av Velocity-Verlet-simulering.

\section{Simulering av satellit}
Satellitens omloppsbana går att finna i Figur~\vref{fig:satellite}.
För att satellitens omloppsbana ska bli stabil behöver $\Delta t$ vara kring
$0.01$.
Detta värde togs fram genom att plotta satellitens bana med olika värden för $\Delta t$.
När ett värde hittats som såg ut att ge stabila banor delades det sedan med 10
för att ge en liten säkerhetsmarginal.

\begin{figure}
\begin{center}
	\includegraphics[width=\textwidth]{../uppg1_orbit.png}
\end{center}
\caption{Satellitens omloppsbana.}
\label{fig:satellite}
\end{figure}

I Figur~\vref{fig:satellite:energy} går det att se hur energin i systemet ändras
över tiden.
I Figur~\vref{fig:satellite:linearmomentum} går det att se hur rörelsemängden i
systemet ändras över tiden.
I Figur~\vref{fig:satellite:angularmomentum} går det att se hur
rörelsemängdsmomentet i systemet ändras över tiden.

\begin{figure}
\begin{center}
	\includegraphics[width=0.8\textwidth]{../uppg1_energy.png}
\end{center}
\caption{Energin för satelliten i omloppsbana.}
\label{fig:satellite:energy}
\end{figure}

\begin{figure}
\begin{center}
	\includegraphics[width=0.8\textwidth]{../uppg1_momentum_linear.png}
\end{center}
\caption{Rörelsemängden för satelliten.}
\label{fig:satellite:linearmomentum}
\end{figure}

\begin{figure}
\begin{center}
	\includegraphics[width=0.8\textwidth]{../uppg1_momentum_angular.png}
\end{center}
\caption{Rörelsemängdsmoment för satelliten.}
\label{fig:satellite:angularmomentum}
\end{figure}


\section{Omloppstiden}
	\subsection{Beräkning av omloppstid}
För att beräkna omloppstiden gör vi om positionen till vinkeln relativt x och
y-axeln, sedan itererar vi igenom den resulterande vektorn och ser när vinkeln
återkommer till samma värde.

Omloppstiden för satelliten är beräknad till: $37$ tidsenheter.

Om initialhastigheten ökas till runt $1.4$ hastighetsenheter lämnar satelliten sin omloppsbana.
Se Figur~\vref{fig:orbit:mutlipleorbits}.
I Figur~\vref{fig:orbit:timeforvelocity} kan vi se att omloppstiden ökar exponensiellt
med initialhastigheten.

\begin{figure}
\begin{center}
	\includegraphics[width=0.8\textwidth]{../uppg2_orbits.png}
\end{center}
\caption{
Omloppsbanor då initialhastigheten förändras hos satelliten.
}
\label{fig:orbit:mutlipleorbits}
\end{figure}

\begin{figure}
\begin{center}
	\includegraphics[width=0.8\textwidth]{../uppg2_energy.png}
\end{center}
\caption{
Som vi kan se så ökar omloppstiden exponensiellt med initialhastigheten.
Som förväntat ökar energin kvadratiskt.
}
\label{fig:orbit:timeforvelocity}
\end{figure}

	\subsection{ISS}
För att undersöka om vår simuleringsmodell fungerar testar vi att simulera
rymdstationen ISS omloppsbana kring jorden med verklig data.

Den data vi fört in i system återfinns i Tabell~\vref{iss:data}.
\begin{table}
\begin{center}
\begin{tabular}{c|c}
	konstant & värde \\
	\hline
	$G$       &  $6.67384 \cdot 10^{-11}$ \\
	$m_J$     &  $5.972 \cdot 10^{24}$ \\
	$m_{\mathrm{ISS}}$ &  $4.5 \cdot 10^5$ \\
	$v_0$     &  $7.7066 \cdot 10^3$ \\
	$h$       &  $4.12 \cdot 10^5$ \\
	$r_J$     &  $6.371 \cdot 10^6$
\end{tabular}
\caption{
Tabell över data använt i simulering av ISS omloppsbana
kringjorden.[wikipedia]}
\label{iss:data}
\end{center}
\end{table}

Den simulerade omloppstiden för ISS är beräknad till ungefär $5650s$ vilket
tycks stämma ganska bra med det riktiga värdet som ligger på $5570s$
[wikipedia].

Med minskande värde på $\Delta t$ närmar sig den simulerade tiden ett stabilt
värde vilket kan ses i Figur~\vref{iss:tider}.
Omloppsbanorna för simuleringarna återfinns i Figur~\vref{iss:banor}.
Banorna ser stabila ut vid $\Delta t = 100s$ men omloppstiden ovan är beräknad
med $\Delta t = 1.5s$.

% TODO: Gör om plotten med olika färger för olika värden på dt
% TODO: Rita jorden i korrekt storlek eller lägg in en korrekt bild av jorden
\begin{figure}
\begin{center}
	\includegraphics[width=0.8\textwidth]{../uppg2_iss.png}
\end{center}
\caption{
	ISS simulerade omloppsbana kring jorden för ett antal värden på $\Delta t$
}
\label{iss:banor}
\end{figure}

\begin{figure}
\begin{center}
	\includegraphics[width=0.8\textwidth]{../uppg2_iss_time.png}
\end{center}
\caption{
	Beräknad omloppstid för ISS med minskande värde på $\Delta t$
}
\label{iss:tider}
\end{figure}

\section{Tvåkroppsproblemet}
Denna uppgift är överhoppad då den är så pass lik nästkommande uppgift.

\section{Solsystemet}
För att testa om simuleringen fungerar för $n$ stycken kroppar så testar vi att
simulera solsystemets inre kroppar; Solen, Merkurius, Venus, Jorden samt Mars.
Data för dessa kroppar återfinns i Tabell~\vref{solarsystem:data}.

\begin{table}
\begin{center}
\begin{tabular}{c|c|c|c}
	Planet    & Massa ($kg$) & Avstånd till Solen (mätt i $m$) & hastighet ($m/s$)\\
	\hline
	Solen     & $1.988435 \cdot 10^{30}$  & $0.0$                      & $0.0$ \\
	Merkurius & $3.3022 \cdot 10^{23}$    & $5.79091 \cdot 10^{10}$    &  $4.787 \cdot 10^4$ \\
	Venus     & $4.869 \cdot 10^{24}$     & $1.08208 \cdot 10^{11}$    &  $3.502 \cdot 10^4$ \\ 
	Jorden    & $5.9721986 \cdot 10^{24}$ & $1.49598261 \cdot 10^{11}$ &  $2.978 \cdot 10^4$ \\
	Mars      & $6.4191 \cdot 10^{23}$    & $2.279391 \cdot 10^{11}$   &  $2.4077 \cdot 10^4$
\end{tabular}
\caption{
	Tabell över de inre planeternas och solens massor, planeternas avstånd till
	solen relativt jordens samt dess hastighet.[wikipedia]
}
\label{solarsystem:data}
\end{center}
\end{table}
	

		\subsection{Stabil bana}
Som nämnts tidigare är ett problem med tidsdiskretiserad simulering, att
resultatet ändras om man justerar tiden mellan varje iteration.
För att en simulering ska bli stabil måste ett tillräckligt lågt $\Delta t$
väljas.
Som går att se i Figur~\vref{fig:solarsystem:large:dt} så kan man få oönskat
resultat detta värde väljs för högt, men om man istället väljer ett
tillräckligt lågt tal så stabiliseras omloppsbanorna, se
Figur~\vref{fig:solarsystem:small:enough:dt}.

Värdet som användes i simuleringarna togs fram genom att ett antal simuleringar
med minskande värden på $\Delta t$ gjordes. 
När planeternas banor såg jämna ut delades värdet på $\Delta t$ med 10 och
detta värde användes sedan i efterföljande simuleringar.

%		\subsection{Planetära omloppsbanor} % svarat på ovan

		\subsection{Validering av simulering}
Vidare går det att undersöka om energin och rörelsemängden
%samt rörelse\-mängds\-momentet
bevaras genom simuleringen för att validera
simuleringsmodellen.

I Figur~\vref{fig:solarsystem:energy} kan vi se den kinetiska energin för de olika kropparna i solsystemet
tillsammans med den potentiella energin samt den totala energin.
Den totala energin domineras totalt av den potentiella energin.

\begin{figure}
\begin{center}
	\includegraphics[width=0.8\textwidth]{../uppg4_orbit_bad}
\end{center}
\caption{Simulerade omloppsbanor med ett för stort värde på $\Delta t$ ($5\cdot 10^5 s$)}
\label{fig:solarsystem:large:dt}
\end{figure}

\begin{figure}
\begin{center}
	\includegraphics[width=0.8\textwidth]{../uppg4_orbit}
\end{center}
\caption{Simulering av omloppsbanor i solsystemet ($\Delta t = 10^3 s$)}
\label{fig:solarsystem:small:enough:dt}
\end{figure}

\begin{figure}
\begin{center}
	\includegraphics[width=0.8\textwidth]{../uppg4_energy}
\end{center}
\caption{Energin under simuleringen i Figur~\vref{fig:solarsystem:small:enough:dt}.}
\label{fig:solarsystem:energy}
\end{figure}

I Figur~\vref{fig:solarsystem:linearmomentum} kan vi se rörelsemängden i systemet.
Diagrammet visar absolutbeloppet av den totala rörelsemängden.
I Figur~\vref{fig:solarsystem:angularmomentum} visas rörelsemängdsmomentet i systemet.
I Figur~\vref{fig:solarsystem:kineticenergy} visas rörelseenergin i systemet.
Eftersom den totala massan totalt domineras av solens massa så är systemets
masscentrum väldigt nära solens masscentrum vilket gör att bidraget till
rörelsemängdsmomentet från solen praktiskt taget är noll.
%Anledningen till att det har ett så ``stort'' värde trots att det borde vara 0 kan vara att det är ett avrundningsfel.
\begin{figure}
\begin{center}
	\includegraphics[width=0.8\textwidth]{../uppg4_momentum}
\end{center}
\caption{Rörelsemängden under simuleringen i Figur~\vref{fig:solarsystem:small:enough:dt}.}
\label{fig:solarsystem:linearmomentum}
\end{figure}

\begin{figure}
\begin{center}
	\includegraphics[width=0.8\textwidth]{../uppg4_angular}
\end{center}
\caption{Rörelsemängdsmomentet under simuleringen i Figur~\vref{fig:solarsystem:small:enough:dt}.}
\label{fig:solarsystem:angularmomentum}
\end{figure}

\begin{figure}
\begin{center}
	\includegraphics[width=0.8\textwidth]{../uppg4_kinetic}
\end{center}
\caption{Rörelseenergi under simuleringen i Figur~\vref{fig:solarsystem:small:enough:dt}.}
\label{fig:solarsystem:kineticenergy}
\end{figure}

		\subsection{Omloppstider}
Mycket går att analysera genom att undersöka energibevarande och så vidare, men
simulerar man verkligheten bör man även undersöka så att simuleringen verkligen
ter sig som verkligheten.
Av denna anledning har vi undersökt tiden för planeternas omloppsbanor i
Tabell~\vref{table:solarsystem:orbit:reality:check}.
Som vi kan se så stämmer våra simulerade tider speciellt bra med
verkligheten.
De avvikelser som förekommer kan förklaras med bland annat den förenklade modell som har använts.
Alla planeterna har startat på en gemensam linje på sitt medelavstånd från solen och med sin medelhastighet.
Som nämns i labbspecen så ger detta cirkelrunda banor istället för de verkliga, elliptiska.


\begin{table}
\begin{center}
\begin{tabular}{c|c|c}
	Planet    & Simulerad omloppstid & Reell omloppstid \\
	\hline
	Merkurius & $87.9745$ &  $87.9691$ \\
	Venus     & $224.699$ &  $224.698$ \\ 
	Jorden    & $365.127$ &  $365.256$ \\
	Mars      & $682.581$ &  $686.970$
\end{tabular}
\caption{
	Tabell över de inre planeternas omloppstider i jorddygn.[wikipedia]
}
\label{table:solarsystem:orbit:reality:check}
\end{center}
\end{table}

\section{Källkod}
	Matlabkoden går att återfinna på:\\
	\url{https://github.com/asheidan/mek} samt
	\url{https://github.com/hyarion/mek}
	

\end{document}
